%%%%%%%%%%%%%%%%%%%%%%%%%%%%%%%%%%%%%%%%%
% Beamer Presentation
% LaTeX Template
% Version 1.0 (10/11/12)
%
% This template has been downloaded from:
% http://www.LaTeXTemplates.com
%
% License:
% CC BY-NC-SA 3.0 (http://creativecommons.org/licenses/by-nc-sa/3.0/)
%
%%%%%%%%%%%%%%%%%%%%%%%%%%%%%%%%%%%%%%%%%

%----------------------------------------------------------------------------------------
%	PACKAGES AND THEMES
%----------------------------------------------------------------------------------------

\documentclass{beamer}
\usepackage[utf8]{inputenc}
\usepackage[portuguese]{babel}
\mode<presentation> {

% The Beamer class comes with a number of default slide themes
% which change the colors and layouts of slides. Below this is a list
% of all the themes, uncomment each in turn to see what they look like.

%\usetheme{default}
%\usetheme{AnnArbor}
%\usetheme{Antibes}
%\usetheme{Bergen}
%\usetheme{Berkeley}
%\usetheme{Berlin}
%\usetheme{Boadilla}
%\usetheme{CambridgeUS}
\usetheme{Copenhagen}
%\usetheme{Darmstadt}
%\usetheme{Dresden}
%\usetheme{Frankfurt}
%\usetheme{Goettingen}
%\usetheme{Hannover}
%*\usetheme{Ilmenau}
%\usetheme{JuanLesPins}
%\usetheme{Luebeck}
%\usetheme{Madrid}
%\usetheme{Malmoe}
%\usetheme{Marburg}
%\usetheme{Montpellier}
%\usetheme{PaloAlto}
%\usetheme{Pittsburgh}
%\usetheme{Rochester}
%*\usetheme{Singapore}
%\usetheme{Szeged}
%\usetheme{Warsaw}

% As well as themes, the Beamer class has a number of color themes
% for any slide theme. Uncomment each of these in turn to see how it
% changes the colors of your current slide theme.

%\usecolortheme{albatross}
%\usecolortheme{beaver}
%\usecolortheme{beetle}
%\usecolortheme{crane}
%\usecolortheme{dolphin}
%\usecolortheme{dove}
%\usecolortheme{fly}
%\usecolortheme{lily}
%\usecolortheme{orchid}
%\usecolortheme{rose}
%\usecolortheme{seagull}
%\usecolortheme{seahorse}
%\usecolortheme{whale}
%\usecolortheme{wolverine}

%\setbeamertemplate{footline} % To remove the footer line in all slides uncomment this line
\setbeamertemplate{footline}[page number] % To replace the footer line in all slides with a simple slide count uncomment this line

%\setbeamertemplate{navigation symbols}{} % To remove the navigation symbols from the bottom of all slides uncomment this line
}

\usepackage{graphicx} % Allows including images
\usepackage{booktabs} % Allows the use of \toprule, \midrule and \bottomrule in tables

%----------------------------------------------------------------------------------------
%	TITLE PAGE
%----------------------------------------------------------------------------------------

\title{Esteganografia em Imagens Médicas} % The short title appears at the bottom of every slide, the full title is only on the title page

\author{Quem é o autor} % Your name
\institute[USP] % Your institution as it will appear on the bottom of every slide, may be shorthand to save space
{
Orientador: Quem? \\ % Your institution for the title page
\medskip
\textit{} % Your email address
}
\date{2015} % Date, can be changed to a custom date

\begin{document}

\begin{frame}
\titlepage % Print the title page as the first slide
\end{frame}

\begin{frame}
\frametitle{Roteiro} % Table of contents slide, comment this block out to remove it
\tableofcontents % Throughout your presentation, if you choose to use \section{} and \subsection{} commands, these will automatically be printed on this slide as an overview of your presentation
\end{frame}

%----------------------------------------------------------------------------------------
%	PRESENTATION SLIDES
%----------------------------------------------------------------------------------------

%------------------------------------------------
\section{Introdução} 
%------------------------------------------------

\begin{frame}
\frametitle{Introdução}
[Deixar implicito objetivos e justificativas...]
\end{frame}

%------------------------------------------------


%------------------------------------------------
\section{Esteganografia}
\subsection{Definição}
\begin{frame}
\frametitle{O que é esteganografia?}

\end{frame}

%------------------------------------------------
\subsection{Origem do termo}
\begin{frame}
\frametitle{Primeiros usos}

\end{frame}

%------------------------------------------------
\subsection{Criptografia $\times$ Esteganografia}
\begin{frame}
\frametitle{Esteganografia é criptografia?}

\end{frame}

%------------------------------------------------
\section{DICOM}
%------------------------------------------------
\subsection{Formato}
\begin{frame}
\frametitle{Descrição do DICOM}

\end{frame}


%------------------------------------------------
\subsection{Falhas}
\begin{frame}
\frametitle{Problemas do DICOM}

\end{frame}

%------------------------------------------------
\section{Técnicas de Esteganografia}
%------------------------------------------------
\subsection{LSB}
\begin{frame}
\frametitle{Low Significative Bits}

\end{frame}
%------------------------------------------------
\subsection{DBAM}
\begin{frame}
\frametitle{Divisão em Blocos e Alteração da Média}

\end{frame}

%------------------------------------------------
\subsection{MAMM}
\begin{frame}
\frametitle{Método de Alteração da Média Modificado}

\end{frame}

%------------------------------------------------
\section{Resultados}
\begin{frame}
\frametitle{Comparação das imagens}

\end{frame}

%------------------------------------------------
\section{Conclusões}
\begin{frame}[fragile] % Need to use the fragile option when verbatim is used in the slide
\frametitle{Conclusões}

\end{frame}

\end{document}
